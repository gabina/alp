\documentclass[12pt, a4paper]{article}
\setlength{\parindent}{0pt}
\usepackage[utf8]{inputenc}
\usepackage[spanish]{babel}
\usepackage{hyperref}
\usepackage{graphicx}
\usepackage{wrapfig}
\usepackage{caption}
\usepackage{subcaption}
\usepackage{multirow} 
\usepackage{ amssymb }
\usepackage{amsmath}

\begin{document} 
\title{Trabajo Práctico Final\\ Análisis de Lenguajes de Programación} 
\author{Bianchi, Gabina Luz} 
\maketitle

\section*{Qué es}
Este trabajo consiste en el desarrollo de la primera versión del programa Méri. Méri es un clasificador automático de poesía en castellano, basándose en la cantidad de versos y los patrones de rima existentes. Actualmente es capaz de identificar:
\begin{itemize}
	\item Cuarteto
	\item Décima
	\item Lira
	\item Octava Real
	\item Redondilla
	\item Serventesio	
\end{itemize}

\section*{Algunos conceptos importantes}
Rima es la repetición de una secuencia de fonemas a partir de la sílaba tónica al final de dos o más versos. Se establece a partir de la última vocal acentuada, incluida ésta.\\
Existen distintas clasificaciones de rimas. Según su timbre, se pueden encontrar dos tipos distintos:
\begin{itemize}
	\item Rima consonante o perfecta: coinciden todos los fonemas a partir de las vocales tónicas. Por ejemplo:
	
	\medskip
	
	\centering{\textit{selva a su amor, que por el verde su\textbf{elo}\\
no ha visto al cazador que con desv\textbf{elo}}}
	\item Rima asonante o imperfecta: coinciden las vocales, pero hay al menos una consonante que no coincide. Por ejemplo:
	
	\medskip
	
	\centering{\textit{Un nombre de mujer, una blanc\textbf{u}r\textbf{a},\\
un cuerpo ya sin cara, la pen\textbf{u}mbr\textbf{a}.}}
\end{itemize}


\section*{Posibles Extensiones} 

\section*{Bibliografía}
 
 En este ejercicio se pide resolver el problema de las gausianas diagonales y paralelas, variando la dimensión de éstas, con el clasificador Naive Bayes utilizando distribuciones normales para aproximar las probabilidades. Por lo tanto, para calcular P(a$|$C), siendo \textit{a} un valor para el atributo \textit{i}, y \textit{C} una clase, se utiliza una distribución normal, con media y desviación estándar calculada por clase y por atributo. En la Figura 1 se grafican los errores porcentuales en test en función de la cantidad de dimensiones, para las gaussianas diagonales y paralelas, utilizando cada uno de los clasificadores vistos hasta ahora.
 




\end{document}