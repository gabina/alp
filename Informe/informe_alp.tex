\documentclass[12pt, a4paper]{article}
\setlength{\parindent}{0pt}
\usepackage[utf8]{inputenc}
\usepackage[spanish]{babel}
\usepackage{hyperref}
\usepackage{graphicx}
\usepackage{wrapfig}
\usepackage{caption}
\usepackage{subcaption}
\usepackage{multirow} 
\usepackage{ amssymb }
\usepackage{amsmath}

\begin{document} 
\title{Trabajo Práctico Final\\ Análisis de Lenguajes de Programación} 
\author{Bianchi, Gabina Luz} 
\maketitle

\section*{Qué es}
Este trabajo consiste en el desarrollo de la primera versión del programa Méri. Méri es un identificador de tipos de poesía en castellano, que se basa en la cantidad de versos y los patrones de rima existentes. Actualmente es capaz de identificar: redondilla, cuarteto, seguidilla, romance, soneto y décima.	
Sin embargo, como se explicará más adelante, se podrían agregar nuevas estructuras poéticas de manera muy sencilla.
\section*{Algunos conceptos importantes}
A fin de poder entender con mayor profundidad el funcionamiento y desarrollo de este proyecto, se presentan algunas definiciones básicas sobre el dominio específico en cuestión: la poesía.
\subsection*{Rima}
Rima es la repetición de una secuencia de fonemas a partir de la sílaba tónica al final de dos o más versos. Se establece a partir de la última vocal acentuada, incluida ésta.\\
Existen distintas clasificaciones de rimas. Según su timbre, se pueden encontrar dos tipos distintos:
\begin{itemize}
	\item Rima consonante o perfecta: coinciden todos los fonemas a partir de las vocales tónicas. Por ejemplo:
	
	\medskip
	
	\begin{center}
		\textit{selva a su amor, que por el verde su\textbf{elo}\\ no ha visto al cazador que con desv\textbf{elo}}
	\end{center}	 

	\item Rima asonante o imperfecta: coinciden las vocales, pero hay al menos una consonante que no coincide\footnote{En este trabajo se considera a la rima consonante como un caso particular de rima asonante, ya que muchos esquemas de rimas no son estrictos en este sentido.}. Por ejemplo:
	
	\medskip
	
	\begin{center}
			\textit{Un nombre de mujer, una blanc\textbf{u}r\textbf{a},\\un cuerpo ya sin cara, la pen\textbf{u}mbr\textbf{a}.}	
	\end{center}

\end{itemize}

\subsection*{Patrones de rima}
A continuación se presentan las características principales de las estructuras poéticas consideradas en el proyecto. Es importante destacar que, en esta versión, el programa no es capaz de identificar cada una de las propiedades, sino que analiza únicamente cantidad de versos y esquemas de rima, dejando para una extensión futura la métrica de cada verso.
\begin{itemize}
	\item Redondilla: estrofa de cuatro versos, normalmente osctasílabos, con patrón de rima consonante \textit{abba}
	\item Cuarteto\footnote{Por lo mencionado anteriormente, en esta versión de Méri, la redondilla y el cuarteto utilizan la misma métrica, ya que el programa no está preparado para diferenciar la cantidad de sílabas de cada verso.} estrofa de cuatro versos, normalmente endecasílabos, con patrón de rima consonante \textit{abba}
	\item Seguidilla: estrofa de arte menor formada por cuatro versos. Los impares, heptasílabos y libres, y los pares, pentasílabos con rima asonante. El patrón de rima es \textit{abcb}.
	\item Romance: estrofa de arte menor formada por cuatro versos octosílabos, con el primero y el tercero libres, y el segundo y cuarto con rima asonante. El patrón de rima es \textit{abcb}.
	\item Soneto: es una composición poética compuesta por catorce versos de arte mayor, endecasílabos en su forma clásica. Admite distintos patrones de rima, según la época y el autor. En este caso se considerará uno de los más utilizados: \textit{abbacddceffegg}\footnote{Basado en libros de sonetos actuales.}.
	\item Décima:  estrofa constituida por diez versos octosílabos. La estructura de rimas es fija en \textit{abbaaccddc}, pero no hay acuerdo sobre si debe ser asonante o consonante. En Méri se considerará asonante.

\end{itemize}

\section*{Instalación y modo de uso}
Para poder utilizar el software primero se deben descargar los archivos fuentes del siguiente repositorio: \url{https://github.com/gabina/alp}, y tener instalado el sistema de paquetes Cabal\footnote{Cabal es un sistema de paquetes de Haskell que permite a los desarrolladores y usuarios distribuír, usar y reutilizar software fácilmente.}. Luego, deben ejecutarse en una terminal los siguientes comandos:
\begin{verbatim}
      $ cabal install
      $ cabal run
\end{verbatim}

Una vez iniciado Méri, se debe elegir la opción correspondiente al tipo de poema que se desea verificar, y luego ingresar el poema. Éste debe ser escrito en una misma línea, separando los versos con una barra (/), e indicando el fin del poema con un asterisco (*). Por ejemplo:
\begin{verbatim}
     soy el que baja al chino de la cuadra/ con lluvia con 
     burbujas en los charcos/ saludando a porteros medios 
     parcos/ y al perro del florista que me ladra/*
\end{verbatim}\footnote{Extracto de un soneto de Pedro Mairal}
\section*{Méri por adentro}
En esta sección se presentarán las ideas principales sobre el diseño y funcionamiento de Méri, detallando cada uno de los módulos involucrados y proveyendo casos de prueba.

\subsection*{Paquete Cabal}
Además de los módulos de Haskell específicos para la resolución del problema en sí, un paquete cabal cuenta con dos archivos más. Por un lado, uno con extensión \textit{.cabal}, el cual contiene metadatos sobre el paquete, incluyendo nombre, versión, descripción, tipo de licencia, dependencias, etcétera. Por el otro, un archivo \textit{Setup.hs}. Es un programa de Haskell de un único módulo que realiza tareas de seteo. En este caso se ha dejado con el seteo estándar que se genera automáticamente al crear un paquete cabal con el comando \textit{cabal init}.  

\subsection*{Common}
En el módulo \textit{Common.hs} se encuentran los tipos de datos utilizados.

\medskip
Para representar los poemas, los versos y las sílabas, se utilizan sinónimos de tipos:
\begin{verbatim}
    type Verse = String
    type Poem = [Verse]
    type Syllable = String
\end{verbatim}

Para trabajar cómodamente con distintas estrcturas poéticas, se definió un tipo de dato \textit{Metric}, con dos posibles constructores, respondiendo a rima asonante o consonante. Además, este tipo de datos contiene un entero indicando la cantidad de versos correspondiente, y un conjunto de conjuntos de enteros donde se indican qué versos deben rimar entre sí.
\begin{verbatim}
 Metric = Asonante Int (Set (Set Int)) | Consonante Int (Set (Set Int))
\end{verbatim}

Por ejemplo, para referirse a una redondilla se debería utilizar:
\begin{verbatim}
    Consonante 4 {{0,3},{1,2}}
\end{verbatim} \footnote{La notación con corchetes no es válida en Haskell. Es a modo ilustrativo.}
Se eligió esta tipo de datos con el objetivo de que una métrica dada tenga una única representación posible\footnote{Por esta razón se obvió el uso de listas. Las listas imponen un orden en sus elementos que en este caso no sería apropiado, ya que se desea que decir \textquotedblleft el verso 0 rima con el 3\textquotedblright  sea equivalente a decir \textquotedblleft el verso 3 rima con el 0\textquotedblright.}. Sin embargo, supongamos que se necesita representar una estructura poética de 5 versos con patrón de rima asonante \textit{abbab}. En ese caso, lo natural sería utilizar:
\begin{verbatim}
    Asonante 5 {{0,3},{1,2,4}}
\end{verbatim}
No obstante, la siguiente sería una representación también válida de la misma estructura poética:
\begin{verbatim}
    Asonante 5 {{0,3},{1,2},{1,4}}
\end{verbatim}
Para (ayudar a) evitar la multiplicidad de representaciones, se creó la función \textit{checkMetric}, que se encarga de chequear que todos los conjuntos sean disjuntos uno a uno y que ningún elemento del conjunto sea mayor o igual a la cantidad de versos. En caso de que la métrica no cumpla estas propiedades, el programa falla, lanzando un error interno. Para lograr esto se utiliza la mónada \textit{Input}, presentada a continuación.
\begin{verbatim}
    data Input a = IN { runInput :: IO (Either String a) }
\end{verbatim}
La mónada \textit{Input} es, en esencia, una mónada \textit{IO}, que además de tener efectos de entrada/salida, devuelve un valor \textit{Either}. \\
La mónada \textit{IO} es necesaria ya que el programa debe informar al usuario el resultado: si el poema ingresado corresponde a la estructura poética elegida o no. Además, en caso de que no corresponda, Méri avisa los fallos que encontró. \\
La mónada \textit{Either String} es muy útil para el manejo de errores. En este caso, los valores correspondientes al constructor \textit{Left} indicarán un error interno, mientras que el constructor \textit{Right} supone el correcto funcionamiento del programa\footnote{Esto no implica que el poema ingresado matchee con la estructura buscada, sino que el análisis se está llevando a cabo sin errores.}. 
\subsection*{Functions}
En el módulo \textit{Functions.hs} se encuentran todas las funciones definidas para el trabajo con poemas y métricas. El algoritmo para separar en sílabas, implementado en la función \textit{syllabifier} está basado en [2].
\subsection*{Options}
En el módulo \textit{Options.hs} se escriben todos los programas correspondientes a los distintos esquemas poéticos. En esencia, asocia cada tipo de poema a una instancia del tipo \textit{Metric}. Allí es donde se deben agregar nuevos esquemas si fuera el caso deseado. 
\subsection*{Parsers}
En el módulo \textit{Parsers.hs} se encuentra el parser capaz de consumir el poema ingresado por el usuario.
\subsection*{Main}
El módulo \textit{Main.hs} es el módulo principal, desde donde se lanza la interacción con el usuario. En caso de agregar un nuevo esquema poético, debería modificarse.
\subsection*{Casos de prueba}
A continuación se presentan casos de prueba exitosos para los distintos esquemas de Méri. También se pueden encontrar en el archivo \textit{tests/ok.txt}.
\begin{verbatim}
    Cuarteto
    
    qué vértigo el abrazo demandante/ los fémures abiertos 
    y el imán/ la pura gravedad que empuja a adán/ al fondo
    de la tierra alucinante/*

    Redondilla

    por juramento y por arcos hay un bar/ vacío con la mesa 
    en la que hablamos/ hace casi dos años y hoy estamos / 
    sin llamarnos sin vernos sin hablar /*    

    Seguidilla
    
    Pues andáis en las palmas/ ángeles santos/que se duerme mi
    niño /tened los ramos/*
    
    Romance
    
    Yo voy como un ciego/ por esos caminos./Siempre pensando en 
    la penita negra/que llevo conmigo./*

    Soneto (este no anda aún porque me falta arreglar una cosita)
    
    profunda en la ciruela está la casa/ la casa que no está y 
    una ciruela/ regresa la voz dulce de la abuela/ otra vez un 
    jardín y una terraza/ el árbol con las ramas acostadas/ la 
    pileta con sol y sombra y siesta/ la mesa todavía medio 
    puesta/ las cigarras cayéndose incendiadas/ los grandes se 
    entregaron al sopor/ quedó el jardín enorme encandilado/ 
    hagan la digestión tengan cuidado/ hermanos aburridos de 
    calor/ trepados al ciruelo rojo oscuro/ escupen los carozos 
    del futuro/*

    Décima
    
    La vida de entre las manos/se nos escurre VELOZ/Se me atraganta
    la voz/de ver como los humanos/perdemos en hechos vanos/el 
    sentido de la esencia/haciendo que la presencia/en este terreno
    hogar/sea un venir, divagar/y perder nuestra existencia/*
           
\end{verbatim}

\subsection*{Nuevos esquemas}

\section*{Posibles Extensiones} 

\section*{Bibliografía}
 \begin{itemize}
 	\item Wikipedia
 	\item A Syllabification Algorithm for Spanish. Heriberto Cuayahuitl [2]
 \end{itemize}




\end{document}